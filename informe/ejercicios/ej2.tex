\subsection{Problema a resolver}
El siguiente ejercicio consiste en exponer un algoritmo capaz de devolver, dado un conjunto de intervalos, el subconjunto máximo de los que no se solapan entre sí. Para ello, llamemos $i_{x}$ al primer elemento del intervalo $x$ y $f_{x}$ al segundo, con $i_{x} \leq f_{x}$. Luego, se pide que el resultado sea uno de los conjuntos $C$ de máximos intervalos posibles que cumpla con la siguiente propiedad: $$\forall m, n \in \mathbb{N}, m \neq n, (v_{m}, v_{n} \in C \Rightarrow \ (f_{v_{m}} < i_{v_{n}}) \vee (f_{v_{n}} < i_{v_{m}})).$$\newline
 Sea cada intervalo la fecha en la que cada profesor del programa de Profesores Visitantes de la FCEyN dictará su curso. Luego, éstos se conforman por una fecha de inicio $i$ y una de fin $f$, donde $i$ es siempre menor o igual a $f$.  Por lo tanto, el algoritmo debe poder tomar las fechas e indicar qué cursos deberían elegirse para maximizar la cantidad de éstos en el ciclo. Para la simplificación del manejo de los datos, éstos se representan con números enteros positivos.\newline
\newline
\textbf {Formatos de entrada y salida:}\newline
\newline
La entrada del algoritmo contiene varias instancias del problema. Cada instancia consta de una línea con el siguiente formato:

$$n\ i_{1}\ f_{1}\ i_{2}\ f_{2}\ ...\ i_{n}\ f_{n}$$


donde \textbf{$n$} es la cantidad de cursos ofrecidos por los profesores (numerados de 1 a n) y los valores \textbf{$[i_{1},f_{1}],\ ...,\ [i_{n},f_{n}]$} representan los días de inicio y fin de cada uno de los n cursos. Todos los datos son enteros positivos. La entrada concluye con una línea comenzada por \# que no debe ser procesada.\newline

La salida debe contener una línea por cada instancia de entrada, donde se listan los números de los cursos elegidos para el ciclo de cursos.\newline
\newline
En lo que sigue, presentaremos dos ejemplos sobre cómo debería comportarse nuestro algoritmo:\newline

\begin{itemize}
\item {\large{\textbf{Ejemplo 1:}}}\newline
En este ejemplo, decidimos develar un caso en el que no se solapara ningún curso. Luego, se puede ver que todos los cursos pueden realizarse en el ciclo de Profesores Visitantes de la FCEyN.
\begin{figure}[H] %[h] Aqui [b] para button [t] para top
\begin{center}
\includegraphics[width=460pt]{../imgs/ejemplo1ej2.png}
\end{center}
\includegraphics[width=90pt]{../imgs/leyendaej2.png}
\caption{Ejemplo 1.}
\end{figure}

\textbf{Formato de entrada:} $$4\ \ 0\ \ 1\ \ 2\ \ 2\ \ 3\ \ 5\ \ 6\ \ 9$$

\textbf{Formato de salida:} $$3\ \ 1\ \ 2\ \ 3\ \ 4$$

\item {\large{\textbf{Ejemplo 2:}}}\newline
En cambio, en el ejemplo que sigue preferimos mostrar un caso en el que algunos cursos se solaparan entre sí dándole al algoritmo la obligación de seleccionar el máximo subconjunto entre ellos.

\begin{figure}[H] %[h] Aqui [b] para button [t] para top
\begin{center}
\includegraphics[width=480pt]{../imgs/ejemplo2ej2.png}
\end{center}
\includegraphics[width=90pt]{../imgs/leyendaej2.png}
\caption{Ejemplo 2.}
\end{figure}

\textbf{Formato de entrada:} $$4\ \ 0\ \ 3\ \ 2\ \ 5\ \ 5\ \ 6\ \ 7\ \ 10$$

\textbf{Formato de salida:} $$3\ \ 1\ \ 3\ \ 4$$

\end{itemize}

\subsection{Resolución coloquial}

Decidimos resolver el problema ordenando los cursos de acuerdo a la fecha de finalización de los mismos por orden creciente. De este modo, se recorrió cada intervalo correspondiente a cada curso y, cuando el $i$-ésimo no se solapaba con el $i$-ésimo-1, se lo agregaba a la solución.\newline
Al analizar el problema a resolver, nos percatamos de que el algoritmo de maximización de intervalos no solapados corresponde a un $algoritmo\ goloso$. El pseudocódigo que describe el algoritmo es el siguiente:\newline

\begin{algorithm}[H]
    \SetAlgoLined
    \caption{Algoritmo Maximizador de la Cantidad De Intervalos No Solapados}
    \KwIn{Cursos $cs$}
    \KwOut{listaDeCursos}
    Lista $seleccionados$ = nuevaLista($\#cs$,$false$)\\
    \eIf{\#cs == 0}{
        \textbf{devolver} $ListaVacia$\\
    }{
        $cs$ = OrdenarPorFin($cs$)\\
        filtrarSolapamientos($cs$, $seleccionados$)\\
        Lista $res$ = filtrar($seleccionado$)\\
        \textbf{devolver} \#res ++ res\\
    }
\end{algorithm}

\begin{algorithm}[H]
    \SetAlgoLined
    \caption{Algoritmo Filtrador de Solapamientos}
    \KwIn{Cursos $cs$}
    \KwOut{listaDeCursos}
    Curso $anterior$ = Primer curso de $CS$\\
    \For{Curso $c$ en $cs$}{
        \If{$c$ no se solapa con anterior}{
            $anterior$ := $c$\\
            $seleccionado_{c.indice} := true$\\
        }
    }
    \textbf{devolver} $res$\\   
\end{algorithm}


donde $Cursos$ es una secuencia conteniendo la información (fecha de inicio y de fin) de los cursos del ciclo, $ordenarPorFin$ es un algoritmo que toma los cursos y los ordena de acuerdo a su fecha de finalización y $no\ se\ solapa$ verifica que la fecha de inicio del intervalo que se quiere agregar no se superponga con la fecha de fin del agregado precedentemente. \textbf{nuevaLista} crea una lista nueva a partir de los parametros de entrada que definen su tamaño y sus valors iniciales. Por otro lado, $filtrar$ recorre la lista de $seleccionados$ y se queda con aquellos cuyo valor es $true$.

\subsection{Demostración de correctitud}

Veamos que, efectivamente, nuestro algoritmo encuentra una solución óptima $S$ cuya cantidad de intervalos es $k$. Para ello, vamos a suponer que tenemos la salida ordenada por la fecha de fin de cada intervalo. Supongamos que existe una solución óptima $S'$ cuya cantidad de intervalos es $n$, dicha solución se encuentra ordenada del mismo modo que $S$.\\
\newline
Sean $S_{1}$ y $S'_{1}$ los primeros intervalos de ambas soluciones. En el caso en el que $f_{S_{1}} \leq f_{S'_{1}}$, se puede reemplazar $S'_{1}$ por $S_{1}$ pues, si $S'_{2}$ no se solapa con $S'_{1}$, tampoco lo hará con $S_{1}$. El caso $f_{S'_{1}} < f_{S_{1}}$ no puede ocurrir pues nuestro algoritmo selecciona el intervalo cuya fecha de fin es la menor.\\
\newline
Por otra parte, si consideramos $S_{2}$ y $S'_{2}$ y por el mismo criterio utilizado anteriormente, si $f_{S_{2}} \leq f_{S'_{2}}$, se puede reemplazar $S'_{2}$ por $S_{2}$ pues, si $S'_{1}$ no se solapa con $S'_{2}$, tampoco se solapará con $S_{2}$. Este mismo razonamiento puede extenderse, con el mismo principio, hasta $S_{k}$.\\
\newline
Tal como mencionado, los intervalos de nuestra solución terminan igual o antes que los de la solución óptima, en particular, el último de ellos. Veamos ahora, qué relación existe entre $k$ y $n$. Para ello, analicemos las diferentes posibilidades:
\begin{itemize}
\item \textbf{$k = n$:} En este caso, la solución proporcionada por nuestro algoritmo tiene la misma cantidad de intervalos que la óptima. Luego, es óptimo.
\item \textbf{$k > n$:} Esta situación no es posible dado que $n$ es la cantidad de intervalos de la solución óptima por lo que $k \leq n$.
\item \textbf{$k < n$:} En este caso, existe un intervalo en $S'_{k+1}$ cuya fecha de fin es mayor a la fecha de fin de $S_{k}$ dado que éste reemplazó a $S'_{k}$. Pero esto resulta absurdo pues este intervalo debería estar también en $S$ como lo estipula nuestro algoritmo.
\end{itemize}

Luego, podemos concluir que $k = n$, siendo, la solución de nuestro algoritmo, la óptima al problema. 

\subsection{Complejidad del algoritmo}

Sea $n$ la cantidad de cursos del ciclo del programa de Profesores Visitantes de la FCEyN. Para analizar la complejidad de nuestro algoritmo, estudiémoslo por partes:
\begin{itemize}
\item En primer lugar, es necesario mencionar que la función \textbf{size} es $\mathcal{O}(1)$\footnote{http://en.cppreference.com/w/cpp/container/vector/size}. Por otra parte, las asignaciones, sumas, restas y otras operaciones básicas se realizan en $\mathcal{O}(1)$ debido a que operamos en un modelo uniforme.
\item La función $filtrarSolapamientos$ contiene un ciclo principal que recorre todos los cursos evaluando que la fecha de inicio del elemento actual sea mayor a la fecha de fin del anterior ($\mathcal{O}(1)$), completando el arreglo $seleccionados$ de acuerdo al resultado obtenido por dicha comparación ($\mathcal{O}(1)$\footnote{http://en.cppreference.com/w/cpp/container/vector/operator\_at}). Dado que dichas operaciones se realizan para cada intervalo del ciclo, la complejidad resulta en $\mathcal{O}(n)$.
\item La función principal 	compara el tamaño del conjunto de intervalos con 0. Si dicha comparación resulta verdadera, se aplica la función \textbf{push$\_$back} sobre la estructura \textbf{vector}, cuya complejidad es $\mathcal{O}(1)$\footnote{http://en.cppreference.com/w/cpp/container/vector/push$\_$back}. Por otro lado, si la comparación es falsa, se realiza \textbf{push$\_$back} sobre los $false$ de forma cíclica recorriendo todos los intervalos del ciclo, siendo esto $\mathcal{O}(n)$ * $\mathcal{O}(1)$ = $\mathcal{O}(n)$.
\begin{itemize}
\item Luego, se aplica la función \textbf{Sort} desde el primer intervalo del conjunto hasta el último, siendo su complejidad $\mathcal{O}(n\ log\ n)$\footnote{http://en.cppreference.com/w/cpp/algorithm/sort}.
\item Posteriormente, se aplica la función $filtrarSolapamientos$, mencionada previamente, cuya complejidad es $\mathcal{O}(n)$.
\item Luego, la función \textbf{push$\_$back} es aplicada nuevamente para $cantSeleccionados$, estando esta acotada por el tamaño de $seleccionados$, cuya complejidad es $\mathcal{O}(n)$.
\item Por último, se evalúa dentro de un ciclo 	si $seleccionados$ en actual es verdadero, en cuyo caso se realiza \textbf{push$\_$back} sobre el que le sigue. Dicho ciclo se realiza para todo el tamaño de $seleccionados$. Luego, la complejidad de dicho ciclo es $\mathcal{O}(n)$.
\end{itemize}
\end{itemize}

Dado que las complejidades dentro de la función se suman entre ellas y las constantes pueden ser omitidas, la complejidad final va a quedar definida por el ciclo principal ($\mathcal{O}(n)$) y el ordenamiento inicial ($\mathcal{O}(n\ log\ n)$). Dado que ambas operaciones se realizan paralelamente, sus complejidades se suman, por lo que la complejidad final es ($\mathcal{O}(n\ log\ n)$) + $\mathcal{O}(n)$, que se encuentra acotada asintóticamente por $\mathcal{O}(n\ log\ n)$. Luego, la complejidad de nuestro algoritmo resulta menor a $\mathcal{O}(n^2)$, que fue lo requerido por la cátedra.

\subsection{Código fuente}

\begin{figure}[H]
\begin{center}
\begin{verbatim}
if (cs.size() == 0) {
    res.push_back(0);
    return res;
} else {
    for (int i = 0; i < cs.size(); ++i)
        seleccionados.push_back(false);
    sort(cs.begin(), cs.end(),customLess);
    filtrarSolapamientos();
    res.push_back(cantSeleccionados);
    for (int i = 0; i < seleccionados.size(); ++i)
        if (seleccionados[i])
            res.push_back(i+1);
    return res;
}
\end{verbatim}
\caption{Ciclo que selecciona los intervalos correspondientes}
\end{center}
\end{figure}

\begin{figure}[H]
\begin{center}
\begin{verbatim}
for(int i = 1; i < cs.size(); ++i) {
    if (cs[i].second.first > anterior.second.second) {
        anterior = cs[i];
        seleccionados[cs[i].first] = true;
        cantSeleccionados++;
    }
}
\end{verbatim}
\caption{Ciclo de la función filtrarSolapamientos}
\end{center}
\end{figure}

\subsection{Instancias posibles}

Para verificar la correctitud de nuestro programa, dispusimos variar estratégicamente las instancias de entrada al ejecutarlo.
\begin{itemize}
\item En primer lugar, decidimos mostrar un caso en el que no se solapara ningún curso. En esta ocasión, puede observarse que, dado que hay una solución óptima, ésta es única. De este modo, se logra verificar que los cursos de la entrada coinciden con los de la salida.\newline
\textbf{Parámetro de entrada:} $$3\ \ 0\ \ 1\ \ 3\ \ 5\ \ 6\ \ 9$$
\textbf{Parámetro de salida:} $$3\ \ 1\ \ 2\ \ 3$$\newline


\begin{figure}[H] %[h] Aqui [b] para button [t] para top
\begin{center}
\includegraphics[width=470pt]{../imgs/instancia4.jpg}
\end{center}
\caption{Instancia posible nº1.}
\end{figure}

\item Por otra parte, probamos el programa para la situación en la que no se ingresa ningún curso. Esto sería, por lo tanto, el caso vacío. Dado que este caso borde es muy significativo, nos pareció importante mencionarlo.\newline
\textbf{Parámetro de entrada:} $$0$$
\textbf{Parámetro de salida:} $$0$$ \newline


\item Otro caso a considerar es en el que todos los cursos finalizan el mismo día (solapándose todos entre sí). En este caso, cualquiera de ellos es solución del problema (pues es la máxima cantidad de intervalos que no se solapan) pero nuestro algoritmo retorna el primero que encuentra de acuerdo al orden que se les impuso mencionado previamente. \newline
\textbf{Parámetro de entrada:}  $$3\ \ 1\ \ 5\ \ 3\ \ 5\ \ 2\ \ 5$$
\textbf{Parámetro de salida:}  $$1\ \ 1$$\newline

\begin{figure}[H] %[h] Aqui [b] para button [t] para top
\begin{center}
\includegraphics[width=490pt]{../imgs/instancia3.jpg}
\end{center}
\caption{Instancia posible nº3.}
\end{figure}


\item Por último, consideramos interesante develar el caso en el que existe al menos un curso que se solapa. Dicho caso sería el más común en el problema a resolver. Este tipo de situaciones tienen más de una solución óptima. En esta oportunidad, hay dos salidas posibles que son solución del problema. Debido a que nuestro algoritmo almacena, en primer lugar, los intervalos cuya fecha de fin es la más baja, dicho intervalo es el prioritario frente a cualquiera que se le solape.\newline
\textbf{Parámetro de entrada:}  $$4\ \ 0\ \ 3\ \ 3\ \ 5\ \ 7\ \ 10\ \ 8\ \ 9$$
\textbf{Parámetro de salida:}  $$2\ \ 1\ \ 4$$ \newline

\begin{figure}[H] %[h] Aqui [b] para button [t] para top
\begin{center}
\includegraphics[width=470pt]{../imgs/instancia1.jpg}
\end{center}
\caption{Instancia posible nº4.}
\end{figure}

\end{itemize}

\subsection{Testing}
Para realizar las pruebas de complejidad, generamos instancias aleatorias de fechas de cursos alterando la cantidad de los mismos. Estas instancias fueron generadas en $C++$ con la función $rand()$ de forma tal a poder acotarlas por 365. Los intervalos se generaron del siguiente modo: a partir de dos números aleatorios entre 0 y 365, se tomó el menor y se lo posicionó como primer elemento del intervalo y al mayor como segundo. Este mismo proceso se realizó para una cantidad de 1000 a 100000 intervalos con los que se ejecutó el programa. De este modo, logramos medir las pruebas de nuestro algoritmo para comprobar que la complejidad correspondiera con la mencionada anteriormente:\newline
\newline
En el primer caso, generamos nuestro script de forma aleatoria sin imponer ningún tipo de norma. En dicho caso, los resultados obtenidos fueron los siguientes:

\begin{figure}[H] %[h] Aqui [b] para button [t] para top
\begin{center}
\includegraphics[width=460pt]{../imgs/graficoej2_aleatorios.png}
\end{center}
\end{figure}

Como puede observarse, la complejidad del algoritmo realizado respeta la cota mencionada para instancias de tipo aleatorio. Veamos qué sucede cuando se toma alguna decisión sobre los valores ingresados como parámetro:\newline
\newline
En el siguiente caso, los intervalos utilizados para evaluar la función fueron realizados con las mismas herramientas referidas anteriormente a diferencia de que éstos no podían solaparse entre sí. La solución obtenida fue la siguiente:
\begin{figure}[H] %[h] Aqui [b] para button [t] para top
\begin{center}
\includegraphics[width=460pt]{../imgs/graficoej2_distintos.png}
\end{center}
\end{figure}
Puede observarse que la complejidad del algoritmo con dichos parámetros permanece igualmente próxima a $\mathcal{O}(n\ log\ n)$ que en el caso anterior. Esto se debe a que el algoritmo realiza, como primer paso, un ordenamiento de los intervalos, teniendo éste el mayor costo en el desarrollo del algoritmo. Veamos qué ocurre en el caso contrario:\newline
\newline
En el caso en el que los cursos se solapan todos entre sí, el comportamiento de la curva que representa la complejidad del algoritmo se mantiene similar a la de la prueba realizada previamente. Esto ocurre por el mismo motivo mencionado en el caso anterior.

\begin{figure}[H] %[h] Aqui [b] para button [t] para top
\begin{center}
\includegraphics[width=460pt]{../imgs/graficoej2_solapados.png}
\end{center}
\end{figure}

A modo de conclusión, podemos afirmar que la complejidad del algoritmo de máxima cantidad de intervalos no solapados es dada principalmente por el ordenamiento que éste realiza en primer lugar para poder aplicarse como $algoritmo\ goloso$. Luego, el contenido de los intervalos no son lo más relevante al referirse a la complejidad del algoritmo.
