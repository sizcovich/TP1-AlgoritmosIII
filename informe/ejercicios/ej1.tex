\subsection{Problema a resolver}
El siguiente ejercicio consiste en hallar una manera de implementar un sistema proporcionado respetando una cota determinada de orden de complejidad. El problema se sitúa en un centro de distribución de correo que recibe paquetes todos los días cuyo destino final es la sede central de la empresa. Para el transporte de los mismos, éstos son cargados a camiones de igual capacidad. El encargado de logística, Pascual, tiene un sistema que utiliza desde hace años para agilizar la carga de los camiones asegurando el uso de una baja cantidad de los mismos para el envío de paquetes al final del día. Dicho sistema consiste en agarrar los paquetes y ubicarlos en algún camión que ya tenga paquetes dentro, eligiendo entre éstos el que menos peso esté cargando hasta ese momento. Si el peso del paquete permite que éste sea cargado en ese camión, se lo ubica allí, sino, se lo incluye en un nuevo camión.\newline
El problema a resolver se basa en escribir un algoritmo que tome los pesos de los paquetes que hay que acomodar e indique cuántos camiones se van a utilizar y cuánto peso se cargará en cada uno de ellos al final del día considerando el sistema de Pascual. Para esto, se respeta el orden de llegada de los paquetes a medida que ingresan. \newline
Las consideraciones a tener en cuenta son que los camiones tienen la misma capacidad de carga, la cantidad disponible de los mismos alcanza para transportar todos los paquetes y que el peso de un paquete no supera la capacidad de carga de un camión. Del mismo modo, el tamaño de los paquetes no es tenido en cuenta. Además, es importante tener en cuenta que las cargas son valores enteros positivos.\newline
\newline
\textbf {Formatos de entrada y salida:}\newline
\newline
La entrada contiene varias instancias del problema. Cada instancia consta de una línea con el siguiente formato:

$$L\ n\ p_{1}\ p_{2}\ ...\ p_{n}$$


donde \textbf{$L$} es el límite de carga de los camiones, \textbf{$n$} es la cantidad de paquetes a acomodar y \textbf{$p_{1}$, ..., $p_{n}$} son los pesos de cada paquete en el orden en el que deben ser almacenados.\newline

La salida debe contener una línea por cada instancia de entrada, con el siguiente formato:

$$k\ c_{1}\ c_{2}\ ...\ c_{n}$$


donde \textbf{$k$} es la cantidad de camiones utilizados y \textbf{$c_{1}$, ..., $c_{k}$} es el peso que se cargó en cada uno de los \textbf{$k$} camiones al final del día.\newline

En lo que sigue, presentaremos dos ejemplos sobre el sistema impulsado por Pascual:
\begin{itemize}
\item {\large{\textbf{Ejemplo 1:}}}\newline

En este ejemplo, decidimos conveniente develar un caso en el que fuera agregado un paquete a un camión ya cargado. Por otro lado, quisimos contrarestarlo insertando un paquete con una carga que superaba la capacidad del camión creado anteriormente. Por último, nos pareció importante mostrar un caso en el que al agregar un nuevo paquete, si bien un nuevo camión había sido creado, éste era colocado en el camión cuya carga fuera la menor.\newline

\textbf{Formato de entrada:}
$$100\ \ 5\ \ 20\ \ 40\ \ 80\ \ 15\ \ 100$$

\begin{figure}[H] %[h] Aqui [b] para button [t] para top
\begin{center}
\includegraphics[width=320pt]{../imgs/ejemplo1.jpg}
\end{center}
\end{figure}

\textbf{Formato de salida:}
$$3\ \ 75\ \ 80\ \ 100$$


\item {\large{\textbf{Ejemplo 2:}}}\newline

En este ejemplo, quisimos mostrar lo que ocurría en caso en el que cada paquete superara el 50\% de la carga disponible en un camión.\newline

\textbf{Formato de entrada:} 
$$150\ \ 3\ \ 80\ \ 75\ \ 82$$

\begin{figure}[H] %[h] Aqui [b] para button [t] para top
\begin{center}
\includegraphics[width=300pt]{../imgs/ejemplo2.jpg}
\end{center}
\end{figure}

\textbf{Formato de salida:}
$$3\ \ 80\ \ 75\ \ 82$$


\end{itemize}
\subsection{Resolución coloquial}
Al analizar el problema a resolver, nos percatamos de que lo más conveniente era implementar la solución en base a un $algoritmo\ goloso$. Dicho algoritmo consiste en la construcción de una solución seleccionando, en cada paso, la mejor alternativa sin considerar las implicancias de ésta.\newline Por otro lado, dicha técnica de diseño algorítmico resulta fácil de implementar, suele ser eficiente y permite construir soluciones razonables.\newline
\newline
En este caso, la resolución basada en un algoritmo goloso nos permitió tomar cada instancia de la entrada y ubicarla en el lugar más conveniente en ese momento. Esto significa que dada la $i-ésima$ caja ingresada en un mismo día, ésta era ubicada en el camión que menor cargado se encontraba en el momento en el que se la agregaba. Para lograr esto, decidimos utilizar como estructura un heap.\newline
\newline
Un heap consiste en una estructura de datos del tipo árbol binario completo con información perteneciente a un tipo de datos cumpliendo con un cierto orden. Un heap máximo tiene la característica de que cada nodo padre tiene un valor mayor que el de cualquiera de sus nodos hijos.\newline
Para adaptar el problema de Pascual al heap, decidimos que cada nodo debía representar la capacidad de carga disponible correspondiente a un camión determinado en un momento dado.\newline
\newline
El pseudocódigo ideado para resolver el problema es el siguiente:\newline

\begin{algorithm}[H]
	\SetAlgoLined
	\caption{Algoritmo de Pascual}
	\KwIn{capacidadCamion, cantidadDePaquetes, pesosDePaquetes}
	\KwOut{cantidadDeCamiones, listaDeCamiones}
	cantidadDeCamiones = 0\\
	actual = 0\\
	camionesCreados = vacio\\
	\While{pesosDePaquetes $\geq$ vacio}{
	\eIf{camionesCreados $\neq$ vacio $\land$ camionMenosCargado(camionesCreados)\footnote{Esta funcion devuelve el camion menos cargado} $\geq$ $pesosDePaquetes_{actual}$}{
		camionActual = camionMenosCargado(camionesCreados)\\
		camionActual = camionActual - $pesosDePaquetes_{actual}$\\
		camionesCreados $\leftarrow$ camionActual
		}{
			camionActual = capacidadCamion - $pesosDePaquetes_{actual}$	\\
			camionesCreados $\leftarrow$ camionActual\\
			cantidadDeCamiones = cantidadDeCamiones + 1\\
		}	
		actual = actual + 1\\
		
	}
\end{algorithm}

\begin{algorithm}[H]
	\SetAlgoLined
	\caption{Algoritmo de Pascual (version Nacho)}
	\KwIn{Paquetes $ps$}
	\KwOut{cantidadDeCamiones, listaDePesos}

	Camiones $ca \leftarrow \{nuevoCamion\}$\\
	cantidadDeCamiones := 1\\
	\lIf{$ps = \emptyset$}{\textbf{devolver} 0, $\emptyset$}\\
	\For{Paquete p $\in$ $ps$}{
		Camion $c$ := camionMenosCargado($ca$)\\
		\eIf{peso($p$) $\leq$ capacidad($c$)}{
			capacidad($c$) $-=$ peso($p$)
		}{
			$ca \leftarrow$ Camion $cNuevo$\\
			capacidad($cNuevo$) $-=$ peso($p$)
			cantidadDeCamiones + 1;
		}
	}
	\textbf{devolver} cantidadDeCamiones, $ca$



\end{algorithm}

\subsection{Demostración de correctitud}
Esta estructura resultó la más adecuada para nuestra implementación dado que 



\subsection{Complejidad del algoritmo}
Tal como requerido, la complejidad temporal del algoritmo debe ser estrictamente menor a $\mathcal{O}(n^2)$.


\subsection{Código fuente}



\subsection{Instancias posibles}



\subsection{Testing}
